%----------------------------------------------------------------------------------------
%	PACKAGES AND OTHER DOCUMENT CONFIGURATIONS
%----------------------------------------------------------------------------------------

\documentclass[sigconf]{acmart}

\usepackage[english]{babel}

\settopmatter{printacmref=false}
\renewcommand\footnotetextcopyrightpermission[1]{}
\pagestyle{plain}

\usepackage{titlesec} % Allows customization of titles
\titleformat{\section}[block]{\large\scshape\leftalign}{\thesection.}{1em}{}
\titleformat{\subsection}[block]{\large}{\thesubsection.}{1em}{}

\usepackage{hyperref}

%----------------------------------------------------------------------------------------
%	TITLE & AUTHORS SECTION
%----------------------------------------------------------------------------------------

\begin{document}

\title{Recognizing emotions with technology and starting discussions}
\subtitle{Research in Emerging Technologies 2017-2018, final paper}
\date{January 2018}

\author{Randall Theuns}
\affiliation{%
    \institution{Amsterdam University of Applied Sciences}
    \department{Software Engineering}
    \streetaddress{Wibautstraat 2-4}
    \city{Amsterdam}
    \postcode{1091GM}
    \country{The Netherlands}
}
\email{randall.theuns@hva.nl}

\author{Max Beije}
\affiliation{%
    \institution{Amsterdam University of Applied Sciences}
    \department{Business IT \& Management}
    \streetaddress{Wibautstraat 2-4}
    \city{Amsterdam}
    \postcode{1091GM}
    \country{The Netherlands}
}
\email{max.beije@hva.nl}

\author{Frank Portengen}
\affiliation{%
    \institution{Amsterdam University of Applied Sciences}
    \department{Business IT \& Management}
    \streetaddress{Wibautstraat 2-4}
    \city{Amsterdam}
    \postcode{1091GM}
    \country{The Netherlands}
}
\email{frank.portengen@hva.nl}

\begin{abstract}
\noindent
According to Waag Society and the Research Group Crossmedia, recent studies have shown that
young adults are hard to reach when it comes down to (cultural) heritage.
Waag Society is researching how cultural heritage institutions can connect to these groups and 
how heritage objects can be relevant to (young) people.
Both parties believe that a better understanding of the emotions people have,
is very important to learn more about the way people value cultural heritage.
Therefore, Waag Society and the Research Group Crossmedia asked students of the HvA to
design an interactive tool that captures young adults’ emotions and enables them to discuss
these emotions with their peers when looking at (cultural) heritage.
\end{abstract}

\maketitle

%----------------------------------------------------------------------------------------
%	ARTICLE CONTENTS
%----------------------------------------------------------------------------------------

\section{Keywords}
Facial recognition, prototype, emotions, heritage, discussion, expressions

%------------------------------------------------
\section{Introduction}
During the minor 'Research in Emerging Technologies', we've chosen the project 'Emotions in Heritage' that
is commissioned by the Research Group Crossmedia (HvA) and the Waag Society. The Waag Society explores
emerging technologies not only related to the internet, but also related to biotechnology and cognitive sciences.
Art and culture often plays a central role in our research as well.

Our project involves systematically collecting the different emotions people experience when they perceive
(cultural) heritage, and use this information to spark meaningful conversations between different parties
by visualizing the different emotions. Cultural heritage could, for example, mean a painting, a building, or
even a tradition. Earlier research regarding emotion recognition has already been conducted. However, these
researches were mostly conducted from a psychological point of view; what emotions are being expressed
and why do young adults express a specific emotion? As such, models to define the emotions already exist.
Currently, there is a lack of instrumentation to capture these emotions and allow young adults to openly discuss
their emotions regarding heritage.

There are many options available to recognize emotions. For example, facial expression recognition, voice
recognition, text recognition, and wireless signals. Initially, we did research regarding these different
methods. We found a paper (REFERENCE HERE) which contained the various recognition methods and also included
the accuracy with which they were able to recognize emotions. Using these measurements, we made the choice to
limit how our project will recognize emotions to the method with the highest accuracy.

%------------------------------------------------

\section{Methods}
Kek2

%------------------------------------------------

\section{Results}
Kek3

%------------------------------------------------

\section{Discussion}
Kek4

%----------------------------------------------------------------------------------------
%	REFERENCE LIST
%----------------------------------------------------------------------------------------

\bibliographystyle{ACM-Reference-Format}
\bibliography{paper}

%----------------------------------------------------------------------------------------

\end{document}
